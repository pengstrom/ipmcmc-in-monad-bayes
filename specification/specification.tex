\documentclass[a4paper, parskip]{scrartcl}
\usepackage[T1]{fontenc}
\usepackage[utf8]{inputenc}
\usepackage[british]{babel}

\addtokomafont{disposition}{\rmfamily}
\addtokomafont{descriptionlabel}{\rmfamily}

\usepackage{amsmath}
\usepackage{amssymb}

\usepackage{todonotes}

\usepackage{booktabs}
\usepackage{hyperref}

\title{Compiler for a Probabilistic Programming Language \todo{Title draft} \\ \normalfont
\Large Bachelor Thesis Specification}

\author{Per Engström \\ \small Subject Reader: Johannes Borgström \\ \small Supervisor:
  Magnus Lundstedt, Precisit AB}

\date{\today}

\begin{document}

\maketitle

\section{Background}
\label{sec:background}

Probabilistic promgramming languages are a paradigm used to model stochastic
processes. It exposes language features to declaratively define models using
random variables and perform inference on them.

A compiler is a tool to transform a program in source-code form into a target,
often machine code, but could also be another high-level language. It requires
knowledge in many areas including formal languages, algorithms and computer
architecture. For a probabilistic language we also need probability theory,
statistics and even methods from artificial intelligence like hidden markov
models to do inference efficiently.

A compiler combines may areas of computer science and gives an opportunity to
delve into Haskell and functional programming to an intermidiate level.

The project will be carried out at the offices of Precisit AB under the
supervision of Magnus Lundstedt.

\todo[inline]{Precisit goals}

A project \emph{Practical probabilistic programming with monads} by Adam Scibior
et al.\ and the paper \emph{Probabilistic Functional Programming in Haskell} by Martin
Erwig and Steve Kollmansberger describes implementing the machinery for writing
in the probablistic style using Haskell. This will give insigt how to leverage
haskell for the inference engine.

Ina another paper called \emph{Lightweight Implementations of Probabilistic
Programming Languages Via Transformational Compilation} David Wingate et al.\
describe a method for adding stochastic abilities to any exsiting language using
transforms and an accompanying Markov chain Monte Carlo inference engine.

\section{Description}
\label{sec:description}

The goal of the project will be implementing a compiler for a probabilistic
programming language. This includes parsing the source code, generating an
executable and implementing the inference engine.

In addition, the parsing of the language will be published as a Haskell package
if no such package exists and there is enough time.

\todo[inline]{Specific goals and tasks}

\todo[inline]{Subjects of interest and how to evaluate}

During the project I'll try to answer the following questions:

\textbf{Is Haskell a suitable language for implementing a probabilistic
language compiler?}

\textbf{How do we test programs under the influence of chance?}

The first question will be evaluated by how idiomatic the resulting
implementation is and if any language-specific hurdles were encountered during
implementation.

The second question will reveal how well different strategies of testing reveal
(and fails to reveal) bugs.

\section{Method}
\label{sec:method}

I will implement a compiler using Haskell, a purely functional language with a
mature support for parsing using parser combinators. It also featues algebraic
data types and pattern matching which facilitates manipulation of abstract
syntax trees.

Some litterature relavant to the project are:
\begin{itemize}
\item \emph{Compilers: Principles,
  Techniques and Tools} by Alfred V. Aho et al.\ Covering compilers in the
  general. 

  \item \emph{The Design and Implementation of Probabilistic Programming
    Languages} by Noah D. Goodman and Andreas Stuhlmüller
    (\url{http://dippl.org}). Covering
    probabilistic languages in particular.
\end{itemize}

  \todo[inline]{More books/articles/websites}

\subsection{Relevant Courses}
\label{sec:relevant_courses}

The courses most relevant is the compilers course and language semantics
course. In addition, the statistics and artificial intelligence courses will
prove valuable. Only two courses have covered Haskell, but most will be learned
during this project.

\begin{description}
  \item[Compiler Design] Pertaining directly to the design of compilers
    and naturally central to this work.

  \item[Semantics of Programming Languages] Important for specifying
    intent of code.

  \item[Artificial Intelligence] Covers for example Hidden Markov Models for
    inference.

  \item[Computer Architecture] Assembly and low-level targets for the compiler.

  %\item[Automata Theory] Methods and concepts for parsing languages.

  \item[Mathematical Statistics] Concepts for probability, statistics and
    inference.

  %\item[Program Design and Data Structures] Intro to Haskell.

  \item[Advanced Functional Programming] Advanced Haskell usage.
\end{description}

\section{Limitations}
\label{sec:limitations}

The most time consuming task of writing a compiler is various optimisations.
This compiler will not be production-grade in terms of stability and features,
but mainly of academic interest.

The implemted language may be reduced to simplify the compiler, i.e.\ only
supporting a subset of a larger language.

\todo[inline]{Define scope}

\section{Time Schedule}
\label{sec:time_schedule}

The project will start Monday 15 January and end Sunday 18 March nominally. A
high-level preliminary project plan follows.

\begin{center}
  \begin{tabular}{rrl}
    \toprule
    \midrule
    \textbf{Week} & \textbf{Dates} & \textbf{Description} \\
    \midrule
    1 & 15--19 Jan & Subject reading and detailing of project plan. \\
    & & Finalize administrative tasks. \\
    & & Begin writing report. \\
    \midrule
    2 & 22--26 Jan & Write background of report. \\
    & & Continue subject reading. \\
    \midrule
    3 & 22--26 Jan & Begin implementation. \\
    \midrule
    4 & 29 Jan--2 Feb & Write method of report. \\
    & & Continue implementation. \\
    \midrule
    5 & 5--9 Feb & Test implementation. \\
    \midrule
    6 & 12--16 Feb & Continue testing implementation. \\
    & & Write results of report. \\
    \midrule
    7 & 19--23 Feb & Write implementation of report. \\
    \midrule
    8 & 26 Feb--2 Mar & Reflect on original questions. \\
    & & \emph{Possible trip home to Värmland.} \\
    \midrule
    9 & 5--9 Mar & Write conclusion of report. \\
    & & Write discussion of report. \\
    & & Write abstract/summary of report. \\
    \midrule
    10 & 12--16 Mar & Finalize report. \\
    & & Prepare for presentations. \\
    \midrule
    \bottomrule
  \end{tabular}
\end{center}

\end{document}
