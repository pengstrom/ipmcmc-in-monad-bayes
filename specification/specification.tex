\documentclass[a4paper, parskip]{scrartcl}
\usepackage[T1]{fontenc}
\usepackage[utf8]{inputenc}
\usepackage[british]{babel}

\addtokomafont{disposition}{\rmfamily}
\addtokomafont{descriptionlabel}{\rmfamily}

\usepackage{amsmath}
\usepackage{amssymb}

\usepackage{todonotes}

\usepackage{booktabs}
\usepackage{hyperref}

\title{Implementation of iPMCMC in Monad-Bayes \\ \normalfont
\Large Bachelor Thesis Specification}

\author{Per Engström \\ \small Subject Reader: Johannes Borgström \\ \small Supervisor:
  Magnus Lundstedt, Precisit AB}

\date{\today}

\begin{document}

\maketitle

\section{Background}
\label{sec:background}

Probabilistic programming languages are a paradigm used to model stochastic
processes. It exposes language features to declaratively define models using
random variables and perform inference on them. There are several methods to
perform the inference of approximating the resulting distribution. This project
will focus on \emph{Markov Chain Monte Carlo} (MCMC) methods and its variations.

Monad-Bayes~\cite{hbayes} is a Haskell DSL for specifying probabilistic models and supplies
several inference methods including MCMC, \emph{Particle Marginal
Metropolis-Hastings}, \emph{Resample-Move Sequential Monte Carlo} and
\emph{Sequential Monte Carlo} but not the more recent \emph{Interacting
Particle Markov Chain Monte Carlo} (iPMCMC) method developed by Rainforth et al.~\cite{rainforth}

The project will be carried out at the offices of Precisit AB under the
supervision of Magnus Lundstedt with the goal of aquiring insights into
probabilistic programming and its inference methods as well as allowing me to
develop skills in the subject.

\section{Description}
\label{sec:description}

The goal of the project would be implementing the iPMCMC in Haskell in the
framework of the Monad-Bayes project. During the project I'll try to answer the following questions:

\textbf{Is the iPMCMC method faster than the other implemented methods?}

\textbf{Is the iPMCMC method more convenient to use with regards to parameters?}

The first question will be evaluted by doing benchmarks on the existing example
models used in the existing benchmarks and the second question by qualitatively
discussing the different parameters of the methods and how sensitive they are.

\section{Method}
\label{sec:method}

I will implement the method using Haskell, a lazy purely functional language, in
the framework of the Monad-Bayes project.

Some litterature relavant to the project are:
\begin{itemize}
  \item \emph{Interacting Particle Markov Chain Monte Carlo} by Rainforth et
      al~\cite{rainforth}. The paper presenting the method.

    \item \emph{Practical Probabilistic Programming with Monads} by
        Ścibior~\cite{hbayes}. The paper presenting the Monad-Bayes
        framework.
\end{itemize}

\subsection{Relevant Courses}
\label{sec:relevant_courses}

The courses most relevant is the artificial intelligence and advanced
functional programmin courses. In addition will
prove valuable. Only two courses have covered Haskell, but most will be learned
during this project.

\begin{description}
  \item[Artificial Intelligence] Covers for example Hidden Markov Models for
    inference.

  \item[Mathematical Statistics] Concepts for probability, statistics and
    inference.

  \item[Advanced Functional Programming] Advanced Haskell usage.
\end{description}

\subsection{Limitations}
\label{sec:limitations}

The original idea was to implement an entire proabilistic language compiler,
but this had the problem of either be too simple to be academically interesting
or to complicated to be suitable for a 15 credits bachelor project. We decided
to limit the scope to only implement the inference engine in an existing
probabilistic framework.

\section{Time Schedule}
\label{sec:time_schedule}

The project will start Monday 15 January and end Sunday 18 March nominally. A
high-level preliminary project plan follows.

\begin{center}
  \begin{tabular}{rrl}
    \toprule
    \midrule
    \textbf{Week} & \textbf{Dates} & \textbf{Description} \\
    \midrule
    1 & 15--19 Jan & Subject reading and detailing of project plan. \\
    & & Finalize administrative tasks. \\
    & & Begin writing report. \\
    \midrule
    2 & 22--26 Jan & Begin writing background of report. \\
    & & Continue subject reading. \\
    & & \emph{Time off friday}. \\
    \midrule
    3 & 29 Jan--2 Feb & Begin writing method of report. \\
    & & Continue implementation. \\
    & & Begin implementation. \\
    & & \emph{Time off monday}. \\
    \midrule
    4 & 5--9 Feb & Test implementation. \\
    & & Continue implementation. \\
    \midrule
    5 & 12--16 Feb & Continue testing implementation. \\
    & & Begin banchmarking. \\
    \midrule
    6 & 19--23 Feb & Begin writing implementation of report. \\
     &  & Continue testing implementation. \\
    & & Begin writing results of report. \\
    \midrule
    7 & 26 Feb--2 Mar & Reflect on original questions. \\
    & & Continue testing implementation. \\
    & & \emph{Vistor and working remotely.} \\
    \midrule
    8 & 5--9 Mar & Begin writing results of report. \\
    & & Begin writing discussion of report. \\
    & & Begin writing abstract/summary of report. \\
    \midrule
    9 & 12--16 Mar & Finalize report. \\
    & & Prepare for presentations. \\
    \midrule
    \bottomrule
  \end{tabular}
\end{center}

\begin{thebibliography}{99}

    \bibitem{hbayes}{Adam Ścibior, Zoubin Ghahramani, and Andrew D. Gordon.
        2015. Practical probabilistic programming with monads. In \emph{Proceedings
        of the 2015 ACM SIGPLAN Symposium on Haskell} (Haskell '15). ACM, New
      York, NY, USA, 165-176.
      DOI=\url{http://dx.doi.org/10.1145/2804302.2804317}}
  \bibitem{rainforth}{Tom Rainforth, Christian A Naesseth, Fredrik Lindsten, Brooks Paige,
      Jan-Willem van de Meent, Arnaud Doucet, and Frank Wood. Interacting
      particle Markov chain Monte Carlo. In \emph{Proceedings of the 33rd
      International Conference on Machine Learning}, volume 48 of JMLR: W\&CP, 2016}
\end{thebibliography}

\end{document}
